\likechapter{Вступ}

Тема ІС для інтерактивнної роботи з логічними моделями є актуальною, оскільки з проведеного дослідження предметної області видно, що для саме цього використання ІС створено не було.

Метою курсового проекту є проектування ІС для подальшої її реалізації в рамці бакалаврського проекту.

Для досягненння данної мети було визначено наступні завдання.

\begin{enumerate}
	\item Проектування модулів для виконання основного функціоналу;
	\item Проектування модулів і БД для підтримки обробки користувачів за для надання звичного функціоналу систем розробки;
	\item Прототипування графічного інтерфейсу для кращого розуміння яка існує вхідна та вихіда інформація за межами основної місії ІС.
\end{enumerate}

Були приведені архітектурні діарами стандарту UML. Проведено дизайн БД для ІС.

При розробці курсового проекту було використанно:

\begin{enumerate}
	\item ОС - Windows 10;
	\item текстовий редактор - LaTeX;
	\item середовище проектуванння  - UML Designed 9.0;
	\item десктоп версія додатку Draw.io.
\end{enumerate}

Робота складається із вступу, трьох розділів, висновків, додатків та переліку джерел посилання. В першому розділі буде дано характеристику предметної області, проведено порівняння існуючих рішень і виявлення їх особливостей. В другому розділі буде обрано методологію проектування ІС, наведено скріншоти функціональної моделі задачі, описано вхідну і вихідну інформацію. В третьому розділі буде наведено загальну схему інформаційних зв’язків і перелік елементів, які будуть використані в інформаційній системі. 